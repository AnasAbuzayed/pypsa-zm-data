\documentclass{article}
% Packages
\usepackage[utf8]{inputenc}

\usepackage{amsmath}
\usepackage{amssymb}
\usepackage{tabularx}
\usepackage[dvipsnames]{xcolor}
\usepackage{graphicx}
\usepackage{xurl}

\usepackage{geometry}
\geometry{a4paper, total={160mm,240mm}, left=25mm, top=20mm}

\usepackage[style=authoryear, maxcitenames = 2, hyperref=true, backref=false, abbreviate=false, maxbibnames=10]{biblatex} %Imports biblatex package
\usepackage{hyperref}
\addbibresource{references.bib}

\title{PyPSA-Earth Model Validation for Zambia}
\author{Albert Sol\`a Vilalta}
\date{August 2023}

\newcommand{\R}{\mathbb{R}}
\newcommand{\N}{\mathbb{N}}
\newcommand{\E}{\mathbb{E}}

\begin{document}

\maketitle



\section{How to edit this document?}

To access this document, you can use this link \url{https://www.overleaf.com/9955214172jymhghfymjrx}. You might need to register in Overleaf if you have not.

After your first edit, please add your name in the authors field (around line 20 above).

\section{Introduction}

This document keeps track of the PyPSA-Earth model validation for Zambia. To begin with, the same sections as in the paper by \cite{Parzen23} presenting the PyPSA-Earth model have been used. We can modify them as we see fit, depending on the specific situation for Zambia.

\textcolor{blue}{[Albert S]: Beyond the sections in the pypsa-earth paper, I've added one on hydro given its importance in the Zambian case.} 


\section{Network Topology and Length}
\label{SEC:NetworkTopologyAndLength}





\section{Electricity Demand}
\label{SEC:ElectricityCDemand}



\section{Solar and Wind Power Potentials}
\label{SEC:SolarAndWindPowerPotentials}


\section{Hydropower}
\label{SEC:Hydropower}



\section{Power Plant Database}
\label{SEC:PowerPlantDatabase}




\section{Zambia 2022 - Dispatch Validation}
\label{SEC:Zambia2022-DispatchValidation}

This section corresponds to section 5.1 in the pypsa-earth paper. It is a validation of the state of the model in a recent year. We could try first with 2022 as it is the last complete year. If need be, we can go further in the past.












\printbibliography

\end{document}
